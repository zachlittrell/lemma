\documentclass{beamer}
%This removes the navigation bar that appears at the bottom of the slides.
%Tis most unnecessary
\setbeamertemplate{navigation symbols}{}
\usetheme{Warsaw}
\title[A Taste of Mathematical Morphology]
      {Mighty Morphin Image Processing}
\subtitle{A Taste of Mathematical Morphology}
\author{Zach Littrell}
\institute{McDaniel College}

%My own macros

%graphics macros
\newcommand{\pic}[2]{
     \begin{array}{l}
      \includegraphics[#1]{#2}
      \end{array}
}
\newcommand{\imagetransform}[3]{
  \begin{figure}
     \includegraphics[#1]{#2}
     \includegraphics[#1]{#3} 
  \end{figure}
}

%text macros
\newcommand{\bad}[1]{\textcolor{red}{#1}}

%basic set operators
\newcommand{\set}[1]{\left\lbrace #1 \right\rbrace}
\newcommand{\buildset}[2]{\set{#1 \mid #2}}
\newcommand{\tuple}[1]{\left\langle #1 \right\rangle}
\newcommand{\intersect}{\cap}
\newcommand{\union}{\cup}
\newcommand{\bigintersect}[1]{\displaystyle\bigcap_{#1}}
\newcommand{\bigunion}[1]{\displaystyle\bigcup_{#1}}
\newcommand{\power}[1]{\mathbb{P}\left(#1\right)}
\newcommand{\integers}{\mathbb{Z}}
\newcommand{\coords}{\integers \times \integers}
\newcommand{\func}[3]{#1:#2\rightarrow#3}
\newcommand{\meet}[1]{\bigwedge_{#1}}
\newcommand{\join}[1]{\bigvee_{#1}}


%morphological operators
\newcommand{\dilate}{\oplus}
\newcommand{\unidilate}{\delta}
\newcommand{\erode}{\ominus}
\newcommand{\unierode}{\epsilon}
\newcommand{\hitmiss}{\circledast}
\newcommand{\partialless}{\leqslant}

\begin{document}
\begin{frame}
  \titlepage
\end{frame}
\section{Nutshell}
\subsection{Nutshell}
%Here, we discuss the history of morphology
%%\subsection{When I'm 64...}
%%\begin{frame}{History}
%%  1964 in America
%%  \begin{figure}
%%     \includegraphics[width=200pt]{./images/The_Fabs.JPG}
%%   \end{figure} 
%%\end{frame}

\subsection{When I'm 64...}
%Initial history
\begin{frame}{History}{1964 onwards}
$\pic{width=100pt}{./images/matheron.jpg}
\pic{width=75pt}{./images/serra.jpg}
\pic{width=100pt}{images/petrograph.png}
$

\begin{itemize} 
  \item Primarily developed by Jean Serra, an engineer
        \footnote{AND he had degrees in psychology/philosophy,
                  Mathematical Geology, and Mathematics},
        and his PhD advisor, Georges Matheron.
  \item Initially developed to detect patterns in images of minerals.
\end{itemize}
\end{frame}

%More modern day
\begin{frame}{History: The Center of Mathematical Morphology}
  \begin{itemize}
    \item In 1968, the Centre de Morphologie Math\'{e}matique 
          was founded, for the further development of Mathematical Morphology.
          $$\pic{width=100pt}{images/cmm.png}
            \pic{width=100pt}{images/directions.png}$$
  \end{itemize}
\end{frame}

%Today!
\begin{frame}{History: Mathematical Morphology Today}
  Just a few places where Mathematical Morphology can be found:
  \begin{itemize}
     \item Mathematica (a few pictures here were made using Mathematica 8!)
     \item 
  \end{itemize}
\end{frame}

%Nutshell!
\begin{frame}{Mathematical Morphology: Nutshell Edition}
  When in Mathematical Morphology, we...
  \begin{itemize}
    \item Want to analyze properties like shapes in an image.
    \item Want to work with images as sets!
  \end{itemize} 
\end{frame}

\section{Binary Morphology}

\subsection{Binary Images}
%Here, we discuss how we convert binary images into matrices
\begin{frame}{Images $\Rightarrow$ Matrix}
\begin{figure}
\centering
$\pic{width=50pt}{images/cross.png}$
$\Rightarrow$
$\begin{bmatrix}
  1 & 0 & 1\\
  0 & 0 & 0\\
  1 & 0 & 1
\end{bmatrix}$
\end{figure}
\end{frame}

%Here, we discuss how we convert binary images into sets
\begin{frame}{Images $\Rightarrow$ Sets}
\begin{figure}
\centering
$\pic{width=50pt}{images/cross.png}  $
$\Rightarrow
\begin{bmatrix}
  0 & 1 & 0\\
  1 & 1 & 1\\
  0 & 1 & 0
\end{bmatrix}
\Rightarrow$
$\set{\begin{aligned}
      &\tuple{0,1},\\
      &\tuple{-1,0},\tuple{0,0},\tuple{1,0},\\
      &\tuple{0,-1}
      \end{aligned}}$
\end{figure}
\end{frame}

%Introducing Cool Hamblen
\begin{frame}{Introducing Cool Hamblen}
For many examples, we'll use simple shapes.

When we don't, I'm going to rely on my assistant: Cool Hamblen.

$\pic{width=100pt}{./images/coolhamblen.png}
\pic{width=100pt}{./images/binarycoolhamblen.png}$
\end{frame}

\subsection{Set Operators}
%Refleciton first:
\begin{frame}{A Gaggle of Set Operators:Reflection}
  For a set $A \subseteq \integers^2$,

           $$-A = \buildset{\tuple{-x,-y}}
                          {\tuple{x,y} \in A}$$

           $$\pic{width=50pt}{./images/coolhamblen_reflected.png}$$
\end{frame}
%
%Now complement
\begin{frame}{A Gaggle of Set Operators: Inversion}
   For a set $A \subseteq \integers^2$,
          
          $$A^c = \buildset{\tuple{x,y}}
                          {\tuple{x,y}\not\in A}$$

          $$\pic{width=50pt}{./images/coolhamblen_complement.png}$$
\end{frame}
%And now translation
\begin{frame}{A Gaggle of Set Operators: Translation}
  For set $A \subseteq \mathbb{Z}^2$, and $\tuple{a,b} \in \mathbb{Z}^2$


    $$A+z = \buildset{\tuple{x+a,y+z}}
                    {\tuple{x,y} \in A}$$
    $$\pic{width=50pt}{./images/coolhamblen_translated.png}$$
\end{frame} 

\subsection{Morphology Primitives}
%Introducing Dilation and Erosion
\begin{frame}{$\dilate$ Dilation and $\erode$ Erosion}
  The backbone of morphology boils down to two functions:
  \begin{itemize}
    \item $\dilate$ Dilation
    \item $\erode$ Erosion
  \end{itemize}
  They are to Mathematical Morphology what Union and Intersection is for
  Set Theory (they are that deceptively awesome).
\end{frame}
%Defining dilation
\begin{frame}{$\dilate$ Dilation: The Intuition Version}
  We'd like to have a sort of 'smearing' operation. 
  We'll call this dream operation \emph{dilation}.
  \footnote{Fun fact: Matheron called this \textbf{dilatation}\\
            More fun fact: Practically no one else does.}
    $$\pic{width=100pt}{images/dilation_example.png}$$
      Black=Original image\\
      Blue=Added regions in dilation\\
      Red=Structuring element at one point during dilation
\end{frame}

\begin{frame}{$\dilate$ Dilation: The Set Version}
  \begin{definition}
    For binary sets $A$ and $B$, we define the \emph{dilation} of $A$ by $B$
    to be:
    $$A \dilate B = \buildset{z \in \mathbb{Z}^2}
                             {-B+z \intersect A \not= \emptyset}$$
  \end{definition}
  
\end{frame}
%Defining erosion
\begin{frame}{$\erode$ Erosion: The Intuition Version}
  We'd like to have a sort of 'shrinking' operation. 
  We'll call this dream operation \emph{erosion}.
    $$\pic{width=100pt}{images/erosion_example.png}$$
      Black=Original image\\
       Blue=Remaining regions after erosion\\
       Red=Structuring element at one point during erosion
 
\end{frame}

\begin{frame}{$\erode$ Erosion: The Set Version}
   \begin{definition}
    For binary sets $A$ and $B$, we define the \emph{erosion} of 
    $A$ by $B$ to be:
    $$A \erode B=\buildset{z \in \mathbb{Z}^2}{B+z \subseteq A}$$
  \end{definition}
 
\end{frame}
%Checking out what happens when we use off-origin structuring elements.
\begin{frame}{Dilating with Funky Structuring Elements (FSEs)}
  What happens if we try to dilate with structuring elements that aren't
  centered at the origin? (Let's find out!)

  Let $A = \pic{width=50pt}{images/binarycoolhamblen.png}$,
      $B = \pic{width=50pt}{images/fse.png}=
           \set{\tuple{100,100}}$

\end{frame}

\begin{frame}{Dilating with FSEs: Results}

$$\pic{width=50pt}{./images/coolhamblen_translated.png}$$

Well, I'll be a monkey's uncle.

\end{frame}

%Revising the definitions of Dilation and Erosion
\begin{frame}{Dilation 'n' Erosion Redux}

  Dilation can be rewritten as
  \begin{itemize}
    \item $A \dilate B = A + B = \buildset{A + b}{b \in B}$
    \item $A \dilate B = \bigunion{b \in B}A + b$
  \end{itemize}

  Erosion can be rewritten as
  \begin{itemize}
    \item $A \erode B = \bigintersect{b \in B}A + (-b)$
  \end{itemize}
  
\end{frame}

%Showing that Dilation and Erosion are duals of each other
\begin{frame}{Dilation and Erosion Turn Out To Be Brothers}
  Erosion and Dilation can actually be written in terms of the other.

  $(A \dilate B)^c = A^c \erode -B$

  $(A \erode C)^c = A^c \dilate -B$
\end{frame}

\subsection{Morphological Functions}

%Confronting the problem that necessitates hit or miss
\begin{frame}{Using Morphology for Pattern Detection}
  I heard a rumor that Cool Hamblen is now sporting a penguin on his
  shirt. 
  $$\pic{width=100pt}{images/tux_coolhamblen.png}$$
\end{frame}

%The first hit with erosion
\begin{frame}{Using Erosion for Pattern Detection}
  Erosion get's us close...
  $$\pic{width=100pt}{images/tux_coolhamblenhit1.png}$$
\end{frame}

%Why that just won't work
\begin{frame}{Using Erosion for Pattern Detection}
$$\pic{width=100pt}{images/tux_coolhamblenhit2.png}$$
Woopsy daisy.
\end{frame}

%Ok, what's a mildly better way?
\begin{frame}{$\hitmiss$ Hit or Miss Transform: The Intuition Version}
  We'd like to specify not only what we want to match, or 'hit,' 
  but also patterns we DON'T want to match, or 'miss.'

  $B_1 = \pic{width=100pt}{images/binarytux.png}$ 
  $B_2 = \pic{width=100pt}{images/binarytuxmiss.png}$
\end{frame}


%Introducing the concept of Hit or Miss transform.
\begin{frame}{$\hitmiss$ Hit or Miss Transform: The Straightforward Version}

  \begin{itemize}
    \item This we can define already as a set.
          \begin{definition}
          For sets $A \in \mathbb{Z}^2$,
                   $B = \tuple{B_1,B_2} \in \mathbb{Z}^2 \times \mathbb{Z}^2$,
          we define the \emph{hit-or-miss transformation} of 
          $A$ by $B$ as
          $$A \hitmiss B = \buildset{z \in \mathbb{Z}^2}
                                    {B_1+z \subseteq A,
                                     B_2+z \subseteq A^c}$$
          \end{definition}
    \item This needs more morphology... $\pic{width=75pt}{images/Walken-Cowbell.jpg}$
  \end{itemize}
\end{frame}

%The morphological definition of Hit or Miss.
\begin{frame}{$\hitmiss$ Hit or Miss Transform: The Morphological Version}
  \begin{itemize}
   \item 
   Recall that $A \erode B_1$ is all points where $B_1$ is contained
        in $A$, and $A^c \erode B_2$ is all points where $B_2$ is contained
        in $A^c$.
    \item 
    So let's revise our definition:
     \begin{definition}
          For sets $A \in \mathbb{Z}^2,
                    B=\tuple{B_1,B_2}\in \power{\mathbb{Z}^2} \times
                                         \power{\mathbb{Z}^2}$,
          we define the \emph{hit-or-miss transformation} of
          $A$ by $B$ as
          $$A \hitmiss B = A \erode B_1 \intersect A^c \erode B_2$$
          or
          $$A \hitmiss B = A \erode B_1 \setminus A \dilate (-B_2)$$ 
        \end{definition}
   \end{itemize}
\end{frame}

%Let's see it in action!
\begin{frame}{$\hitmiss$ Using Hit or Miss Transform for Pattern Detection}
  $\pic{width=75pt}{images/tux_coolhamblenhitmiss.png} 
   \Rightarrow \pic{width=75pt}{images/tux_coolhamblenhitmisstransformed.png}
   \Rightarrow \pic{width=75pt}{images/tux_coolhamblenhitmisstransformeddilate.png}
$


\end{frame}
%Now some more morphological functions
%Starting with Boundary Extraction
\begin{frame}{A Gaggle of  Binary Morphological Functions:\\
              Boundary Extraction}
We can extract the boundary using the equation:
$$A - (A \erode B)$$

$\pic{width=75pt}{images/binarycoolhamblen.png}
 -
 \pic{width=75pt}{images/coolhamblen_eroded.png}
 =
 \pic{width=75pt}{images/coolhamblen_boundary.png}$

\end{frame}
%The region filling/bucket algorithm
\begin{frame}{A Gaggle of Binary Morpophological Functions:\\
              Region Filling:}
  We can fill up a region using the following recursive equation:
  $$X_k=(X_{k-1}\dilate B)-A$$
  where $X_0 = \set{p}$, for some point $p$ in the region to be filled.

  So, let's fill up the top triangle in this image: 
  $\pic{width=20pt}{images/triangles.png}$

$
    \begin{array}{c|c|c|c|c|c}
     X_1 & X_2 & X_3 & X_4 & X_4 \union A\\\hline
     \pic{width=20pt}{images/triangles1.png}
     &\pic{width=20pt}{images/triangles2.png}
     &\pic{width=20pt}{images/triangles3.png}
     &\pic{width=20pt}{images/triangles4.png}
     &\pic{width=20pt}{images/trianglesfilled.png}
   \end{array}$
 

\end{frame}

%The Connected Component Extraction algorithm%
\begin{frame}{A Gaggle of Binary Morphological Functions:\\
              Connected Component Extraction:}
   We can extract an individual component of an image using the recursive
   equation:
   $$X_k = (X_{k-1} \dilate B) \intersect A$$
     where $X_0 = \set{p}$, for some point $p$ in the component

  Man, I'd sure love to extract that triangle from this image: 
$
   \pic{width=20pt}{images/shapes.png}$

   $\begin{array}{c|c|c|c|c|c}
     X_1 & X_2 & X_3 & X_4 & X_5 & X_6\\\hline
     \pic{width=20pt}{images/shapes1.png}
     &\pic{width=20pt}{images/shapes2.png}
     &\pic{width=20pt}{images/shapes3.png}
     &\pic{width=20pt}{images/shapes4.png}
     &\pic{width=20pt}{images/shapes5.png}
     &\pic{width=20pt}{images/shapes6.png}
   \end{array}$

\end{frame}
              
\section{Grayscale Morphology}
\subsection{Attempt \#1}
%Describing how the problem changes when we move to grayscale
\begin{frame}{What's New With Grayscale?}
  \begin{itemize}
    \item Pixels can now be \textbf{between} 0 and 1.
    \item So our pixels now live in $\mathbb{Z}^3$, namely
          they take the form of $\tuple{x,y,value}$.
    \item Can't be nearly as cheatsy as we were with binary images.
  \end{itemize}
\end{frame}

%Attempting at defining grayscale dilation with Minkowski sum
\begin{frame}{Operation Minkowski: Attempt \#1 at Dilation}
  The Minkowski Sum definition of dilation ought to translate pretty
        well to $\mathbb{Z}^3$ 

  Let $A = \set{\tuple{0,0,0},\tuple{0,1,0}}$

        Let $B = \set{\tuple{1,1,1},\tuple{1,0,0}}$

        $A \dilate B = \set{\bad{\tuple{1,1,1}},
                            \tuple{1,2,1},
                            \tuple{1,0,0},
                            \bad{\tuple{1,1,0}}}$

  Everything okie dokie artichokie?
\end{frame}

\subsection{Attempt \#2}

%Setting up how Dilation should work
\begin{frame}{Operation Supremum: Attempt \#2 at Dilation}
  Given a grayscale image like this
  $$\pic{width=50pt}{images/grayscale_tiles.png}$$  
  Let's examine what should happen at a point
  $$\pic{width=50pt}{images/grayscale_tiles_point.png}$$
\end{frame}

%Giving the intuitive definition
\begin{frame}{Operation Supremum: Intuition at Dilation}
  Using our handy dandy cross structuring element, we'll gather values from
  our neighborhood.
  $$\pic{width=50pt}{images/grayscale_tiles_neighborhood.png}$$  
  And we pick the \emph{largest} (i.e. closest to white) value.
  $$\pic{width=50pt}{images/grayscale_tiles_point_replacement.png}$$
\end{frame}

%Defining dilation in terms of functions
\begin{frame}{$\dilate$ Dilation}
\begin{definition}
  For functions $f,g: \mathbb{Z}\times\mathbb{Z}\rightarrow [0,1]$,
  we define the $dilation$ of $A$ by $B$ as
  $$(A \dilate B)(x,y) = \displaystyle\sup_{\tuple{x',y'} \in \integers^2}
                                      \set{f(x-x',y-y') + g(x',y')}$$
\end{definition}
  $$\pic{width=50pt}{images/grayscale_tiles.png}
    \dilate \pic{width=50pt}{images/cross_inverted.png}
     = \pic{width=50pt}{images/grayscale_tiles_dilated.png}$$
\end{frame}

%Of course we gotta do the same for erosion.
\begin{frame}{Operation Infimum: Intuition at Erosion}
  Once again, we observe the neighborhood formed by our structuring element
  at a point.
  $$\pic{width=50pt}{images/grayscale_tiles_neighborhood2.png}$$
  This time, however, we take the \emph{smallest} (i.e. closest to black)
  value.
  $$\pic{width=50pt}{images/grayscale_tiles_point_replacement2.png}$$
\end{frame}

%Defining Erosion
\begin{frame}{$\erode$ Erosion}
\begin{definition}
  For functions $f,g: \mathbb{Z}\times\mathbb{Z}\rightarrow [0,1]$,
  we define the $erosion$ of $A$ by $B$ as
  $$(A \erode B)(x,y) = \displaystyle\inf_{\tuple{x',y'} \in \mathbb{Z}}
                                      \set{f(x+x',y+y') - g(x',y')}$$
\end{definition}
$$\pic{width=50pt}{images/grayscale_tiles.png}
  \erode \pic{width=50pt}{images/cross_inverted.png}
  = \pic{width=50pt}{images/grayscale_tiles_eroded.png}$$
\end{frame}

\section{Generalized Morphology}
\subsection{The Hunt is on...}
%Describing our basic necessities for generalized morphology
\begin{frame}{What do we want?}
  Things we need:
  \begin{itemize}
    \item We need generalized forms of dilation and erosion.
    \item Ought to work for both Binary and Grayscale morphology.
  \end{itemize} 
\end{frame}

%Reintroducing tools needed for general morphology
\subsection{Lattices}
\begin{frame}{Complete Lattices}
  \begin{definition}
    Let $L$ be a partially ordered set such that 
    every subset of $L$ has a supremum and infimum. 
    We call $L$ a \emph{complete lattice}.
  \end{definition}
  Note, for 
\end{frame}
%Defining dilation for complete lattices
\begin{frame}{$\unidilate$ Dilation on Complete Lattices}
   Let $(L, \partialless)$ be a complete lattice, and $\set{X_i}$ be
   a collection of elements from $L$.
   \begin{definition}
     Let $\func{\unidilate}{L}{L}$ be a function such that
       $$\join{i} \unidilate(X_i) = 
         \unidilate\left(\join{i}X_i\right)$$
     We call $\unidilate$ a \emph{dilation}.
   \end{definition}
\end{frame}

%Defining dilation for complete lattices
\begin{frame}{$\unierode$ Erosion on Complete Lattices}
   Let $(L, \partialless)$ be a complete lattice, and $\set{X_i}$ be
   a collection of elements from $L$.
   \begin{definition}
     Let $\func{\unierode}{L}{L}$ be a function such that
       $$\meet{i} \unierode(X_i) = 
         \unierode\left(\meet{i}X_i\right)$$
     We call $\unierode$ an \emph{erosion}.
   \end{definition}
\end{frame}

\subsection{Getting back our old morphologies}

%Considering the case of the powerset of our plane
\begin{frame}{$\power{\integers^2}$, a Complete Lattice}
Consider the powerset of $\integers^2$.
Let $\partialless$ be $\subseteq$.

Conveniently, $\meet{} = \bigintersect{}$, and $\join{} = \bigunion{}$.

Bingo, bango, we have binary morphology again.
\end{frame}

%Consdering the case of grayscale functions
\begin{frame}{Grayscale functions}
Consider the set of all functions $\func{f}{\coords}{\lbrack 0,1 \rbrack}$.

Let $f,g$ be two such functions.

We say $f \partialless g$ if 
$\forall \tuple{x,y} \in \coords, f(x,y) \leq g(x,y)$

Oingo, boingo, we have grayscale morphology again.


\end{frame}
\end{document}
